% Options for packages loaded elsewhere
\PassOptionsToPackage{unicode}{hyperref}
\PassOptionsToPackage{hyphens}{url}
\PassOptionsToPackage{dvipsnames,svgnames,x11names}{xcolor}
%
\documentclass[
  letterpaper,
  DIV=11,
  numbers=noendperiod]{scrartcl}

\usepackage{amsmath,amssymb}
\usepackage{iftex}
\ifPDFTeX
  \usepackage[T1]{fontenc}
  \usepackage[utf8]{inputenc}
  \usepackage{textcomp} % provide euro and other symbols
\else % if luatex or xetex
  \usepackage{unicode-math}
  \defaultfontfeatures{Scale=MatchLowercase}
  \defaultfontfeatures[\rmfamily]{Ligatures=TeX,Scale=1}
\fi
\usepackage{lmodern}
\ifPDFTeX\else  
    % xetex/luatex font selection
\fi
% Use upquote if available, for straight quotes in verbatim environments
\IfFileExists{upquote.sty}{\usepackage{upquote}}{}
\IfFileExists{microtype.sty}{% use microtype if available
  \usepackage[]{microtype}
  \UseMicrotypeSet[protrusion]{basicmath} % disable protrusion for tt fonts
}{}
\makeatletter
\@ifundefined{KOMAClassName}{% if non-KOMA class
  \IfFileExists{parskip.sty}{%
    \usepackage{parskip}
  }{% else
    \setlength{\parindent}{0pt}
    \setlength{\parskip}{6pt plus 2pt minus 1pt}}
}{% if KOMA class
  \KOMAoptions{parskip=half}}
\makeatother
\usepackage{xcolor}
\setlength{\emergencystretch}{3em} % prevent overfull lines
\setcounter{secnumdepth}{-\maxdimen} % remove section numbering
% Make \paragraph and \subparagraph free-standing
\ifx\paragraph\undefined\else
  \let\oldparagraph\paragraph
  \renewcommand{\paragraph}[1]{\oldparagraph{#1}\mbox{}}
\fi
\ifx\subparagraph\undefined\else
  \let\oldsubparagraph\subparagraph
  \renewcommand{\subparagraph}[1]{\oldsubparagraph{#1}\mbox{}}
\fi


\providecommand{\tightlist}{%
  \setlength{\itemsep}{0pt}\setlength{\parskip}{0pt}}\usepackage{longtable,booktabs,array}
\usepackage{calc} % for calculating minipage widths
% Correct order of tables after \paragraph or \subparagraph
\usepackage{etoolbox}
\makeatletter
\patchcmd\longtable{\par}{\if@noskipsec\mbox{}\fi\par}{}{}
\makeatother
% Allow footnotes in longtable head/foot
\IfFileExists{footnotehyper.sty}{\usepackage{footnotehyper}}{\usepackage{footnote}}
\makesavenoteenv{longtable}
\usepackage{graphicx}
\makeatletter
\def\maxwidth{\ifdim\Gin@nat@width>\linewidth\linewidth\else\Gin@nat@width\fi}
\def\maxheight{\ifdim\Gin@nat@height>\textheight\textheight\else\Gin@nat@height\fi}
\makeatother
% Scale images if necessary, so that they will not overflow the page
% margins by default, and it is still possible to overwrite the defaults
% using explicit options in \includegraphics[width, height, ...]{}
\setkeys{Gin}{width=\maxwidth,height=\maxheight,keepaspectratio}
% Set default figure placement to htbp
\makeatletter
\def\fps@figure{htbp}
\makeatother

\KOMAoption{captions}{tableheading}
\makeatletter
\@ifpackageloaded{caption}{}{\usepackage{caption}}
\AtBeginDocument{%
\ifdefined\contentsname
  \renewcommand*\contentsname{Table of contents}
\else
  \newcommand\contentsname{Table of contents}
\fi
\ifdefined\listfigurename
  \renewcommand*\listfigurename{List of Figures}
\else
  \newcommand\listfigurename{List of Figures}
\fi
\ifdefined\listtablename
  \renewcommand*\listtablename{List of Tables}
\else
  \newcommand\listtablename{List of Tables}
\fi
\ifdefined\figurename
  \renewcommand*\figurename{Figure}
\else
  \newcommand\figurename{Figure}
\fi
\ifdefined\tablename
  \renewcommand*\tablename{Table}
\else
  \newcommand\tablename{Table}
\fi
}
\@ifpackageloaded{float}{}{\usepackage{float}}
\floatstyle{ruled}
\@ifundefined{c@chapter}{\newfloat{codelisting}{h}{lop}}{\newfloat{codelisting}{h}{lop}[chapter]}
\floatname{codelisting}{Listing}
\newcommand*\listoflistings{\listof{codelisting}{List of Listings}}
\makeatother
\makeatletter
\makeatother
\makeatletter
\@ifpackageloaded{caption}{}{\usepackage{caption}}
\@ifpackageloaded{subcaption}{}{\usepackage{subcaption}}
\makeatother
\ifLuaTeX
  \usepackage{selnolig}  % disable illegal ligatures
\fi
\usepackage{bookmark}

\IfFileExists{xurl.sty}{\usepackage{xurl}}{} % add URL line breaks if available
\urlstyle{same} % disable monospaced font for URLs
\hypersetup{
  pdftitle={What's Repetition?},
  colorlinks=true,
  linkcolor={blue},
  filecolor={Maroon},
  citecolor={Blue},
  urlcolor={Blue},
  pdfcreator={LaTeX via pandoc}}

\title{What's Repetition?}
\author{}
\date{}

\begin{document}
\maketitle

\textbf{Learning Objectives}

\begin{itemize}
\tightlist
\item
  Understand the concept of repetition in programming and its role in
  automating tasks
\item
  Identify the two main types of loops in Python: for loops and while
  loops
\item
  Recognize the benefits of using repetition for processing large
  amounts of data efficiently
\end{itemize}

\textbf{Introduction}

Having covered branching and conditionals in previous lessons, this
lesson introduces repetition as another fundamental programming concept.
It sets the stage for the following lessons which will delve into
specific loop constructs in Python, such as for loops and while loops.

\textbf{Executing code multiple times} Repetition in programming refers
to the ability to execute a block of code multiple times. This is a
powerful feature that allows programmers to automate repetitive tasks
and process large amounts of data efficiently. Instead of manually
writing the same code over and over, repetition constructs enable the
code to be executed as many times as needed.

\textbf{Uses loops (for, while)} In Python, repetition is achieved
through the use of loops. The two main types of loops are for loops and
while loops. For loops are used when the number of iterations is known
in advance, such as iterating over a sequence of elements. While loops,
on the other hand, are used when the number of iterations is not known
and depends on a certain condition being met.

\textbf{Like daily weather checks} A real-world analogy for repetition
is performing daily weather checks. Just as we might check the weather
forecast every morning to plan our day, a program can use repetition to
perform a task or check a condition repeatedly. This could involve
retrieving data from a weather API, processing it, and updating a
display or sending notifications based on the current weather
conditions.

\textbf{Automates repetitive tasks} One of the key benefits of
repetition in programming is its ability to automate repetitive tasks.
Instead of manually performing the same action over and over, loops
allow us to write code once and have it execute multiple times. This
saves time, reduces the risk of errors, and makes our code more concise
and maintainable.

\textbf{Efficient for data processing} Repetition is particularly useful
when it comes to processing large amounts of data. Whether it's
iterating over a list of items, reading data from a file, or querying a
database, loops enable us to efficiently process and manipulate data. By
using repetition constructs, we can perform operations on each element
of a dataset without having to write separate code for each individual
item.

\textbf{Key Takeaways}

\begin{itemize}
\tightlist
\item
  Repetition in programming allows executing a block of code multiple
  times
\item
  Loops (for and while) are used to achieve repetition in Python
\item
  Repetition automates repetitive tasks, saving time and reducing errors
\item
  Loops are efficient for processing and manipulating large datasets
\item
  Repetition is a fundamental programming concept that enables powerful
  automation
\end{itemize}

\textbf{Quick Quiz}

\begin{enumerate}
\def\labelenumi{\arabic{enumi}.}
\item
  What is the main benefit of using repetition in programming? Answer: b
\item
  Which type of loop is used when the number of iterations is known in
  advance? Answer: a
\end{enumerate}

\textbf{Additional Resources}

\begin{itemize}
\tightlist
\item
  Python Loops Tutorial:
  https://www.tutorialspoint.com/python/python\_loops.htm
\item
  Real Python - For Loops in Python:
  https://realpython.com/python-for-loop/
\item
  Python While Loops Tutorial with Examples:
  https://www.programiz.com/python-programming/while-loop
\end{itemize}

\emph{Created on: 2024-08-05}



\end{document}
